\PassOptionsToPackage{ELEC}{aaltologo}
\documentclass[logo=bluequo,normaltitle]{aaltoslides}
%\documentclass{aaltoslides} % DEFAULT
%\documentclass[first=purple,second=lgreen,logo=bquo,normaltitle,nofoot]{aaltoslides} % SOME OPTION EXAMPLES

\usepackage[utf8]{inputenc}
\usepackage[T1]{fontenc}
\usepackage{graphicx}
\usepackage{amssymb,amsmath}
\usepackage{url}
\usepackage{lastpage}
\usepackage{epstopdf}
%\usepackage[pdfpagemode=None,colorlinks=true,urlcolor=red, linkcolor=black,citecolor=black,pdfstartview=FitH]{hyperref}
\usepackage{mdframed}
\usepackage{caption}
\usepackage{apacite}
\usepackage{tikz}
\usetikzlibrary{positioning,shapes,shadows,arrows}
\tikzset{
  every overlay node/.style={
    %draw=black,fill=white,rounded corners,
    anchor=north west, inner sep=0pt,
  },
  thick/.style=      {line width=0.3mm},
}
\def\tikzoverlay{%
   \tikz[remember picture, overlay]\node[every overlay node]
}%

%%%%%%%%%%%%%%%The functions are collected here %%%%%%%%%%%%%%%%%%%%%%%%%%%%%%%%%%%%%%%
%%%% This for easy placemnet of figure inside a block
\newcommand{\putfig}[3][1.0]{
    \begin{block}{#3}
        \includegraphics[width=#1\linewidth,height=#1\textheight,keepaspectratio]{#2}
    \end{block}
}

%This is for a box 
\newcommand{\tbox}[3][0.3]{
    \tikzoverlay (n1) at (#2) {%
        \begin{tikzpicture}[every node/.style={inner sep=0.5ex,outer sep=0}]
            \tikzstyle{item}=[rectangle, draw=black, rounded corners, 
                    fill=secondarycolor!10, drop shadow,
                    text centered, anchor=north, text=black] 
                    \node (thenode)[item, rectangle]{
                        \begin{minipage}{#1\linewidth} #3 \end{minipage}};
        \end{tikzpicture}
    };
}
%%%%%%%%%%%%%%%%%%%%%%%%%%%%%%%%%%%%%%%%%%%%%%%%%%%%%%%%%%%%%%%%%%%%%%%%%%%%%%
%This is for a image. For titled image, use putfig 
\newcommand{\timage}[3][0.3]{
    \tikzoverlay (n1) at (#2) {%
        \begin{tikzpicture}[every node/.style={inner sep=0.5ex,outer sep=0}]
            \tikzstyle{item}=[rectangle, draw=black, rounded corners, 
                    fill=secondarycolor!10, drop shadow,
                    text centered, anchor=north, text=white] 
                    \node (thenode)[item, rectangle]{
                        \includegraphics[width=#1\textwidth,height=#1\textheight,keepaspectratio]{#3}};
        \end{tikzpicture}
    };
}


%Ellipse
\newcommand{\tellipse}[4]{
    \begin{tikzpicture}[overlay]
        \draw[secondarycolor,thick] (#1,#2) ellipse (#3 and #4);
    \end{tikzpicture}
}

%Arrow
\newcommand{\tarrow}[2]{
    \begin{tikzpicture}[overlay]
        \draw[secondarycolor,thick,->] (#1) -- (#2);
    \end{tikzpicture}
}



%%%% To insert lecture date
\newcommand{\lectdate}{18.12.2019}
%\newcommand{\lectdate}{\today}
\newcommand{\slidetitle}{Modular BAG configuration}
%%%%

\title{\slidetitle}

\author[Marko Kosunen]{Marko Kosunen}
\institute[MNT]{Department of Micro and Nanosciences\\
Aalto University, School of Electrical Engineering\\marko.kosunen@aalto.fi}

\aaltofootertext{}{\lectdate}{\arabic{page}/\pageref{LastPage}\ }

\date{\lectdate}

\begin{document}

%%%%%%%%%%%%%%%%%%%%%%%%%%%%%%%%%%%%%%%%%%%%%%%%%%%%%%%%%%%%%%%%%%%%%%%%%%%%%%%%%%%%%%%%%%%%%
% Generates the titleframe
\aaltotitleframe
%%%%%%%%%%%%%%%%%%%%%%%%%%%%%%%%%%%%%%%%%%%%%%%%%%%%%%%%%%%%%%%%%%%%%%%%%%%%%%%%%%%%%%%%%%%%%



%\AtBeginSection[]
%{
%%%Outlineframe
%    \begin{frame}{Outline}
%        \tableofcontents[currentsection]
%\end{frame}
%}
%\section*{Outline}
%\begin{frame}[t]
%        \tableofcontents
%\end{frame}


%%%%%%%%%%%%%%%%%%%%%%%%%%%%%%%%%%%%%%%%%%
\begin{frame}[t]
    \frametitle{Outline}
    \begin{itemize}
        \item Programmatic design environment
        \item Modular BAG design environment
        \item BAG\_technology\_definition
        \item Design module
        \item BAG-ecd and bag\_design classes
        \item Generator demo
    \end{itemize}
\end{frame}


%%%%%%%%%%%%%%%%%%%%%%%%%%%%%%%%%%%%%%%%%%
\begin{frame}[t]
    \frametitle{Programmatic design environment}
    \centering
    \resizebox{0.75\textwidth}{!}{%
     \begin{tikzpicture}[every node/.style={inner sep=0ex,outer sep=0}]
\tikzstyle{item}=[draw=black, rounded corners, 
                fill=secondarycolor!10, drop shadow,
                align=center, anchor=center, text=black, ellipse, minimum
                width=3cm,
            minimum height=2cm] 
\node (n1) [item]{Design and\\verification};
\node (ntext) [above = 0.2 cm of n1.north, rectangle, align=center,
text=black, inner sep=1ex, outer sep=1ex]  {Domain of
programmatic abstraction };
\node[ anchor=north, below =2cm of n1.south] (n2) [item]{Implementation\\tools};
\node[ below = 2cm of n2] (n3) [item]{Implementation\\control};
\node[ anchor=north, below left = 2cm and 5cm of n1.south ] (n4) [item]{Digital\\design};
\node[ anchor=north, below right = 2cm and 5cm of n1.south ] (n5)
[item]{Analog\\design};
\draw[>=latex,primarycolor, very thick,<->] (n1.south) to [out=270,in=90]
(n2.north);
\draw[>=latex,primarycolor,very thick,<->] (n2.south) to [out=270,in=90]
(n3.north);
\draw[>=latex,primarycolor,very thick,<->] (n1.west) to [out=180,in=90] (n4.north);
\draw[>=latex,primarycolor,very thick,<->] (n1.east) to [out=0,in=90] (n5.north);
\draw[>=latex,primarycolor,very thick,<->] (n4.east) to [out=0,in=180]
(n2.west);
\draw[>=latex,primarycolor,very thick,<->] (n2.east) to [out=0,in=180]
(n5.west);
\filldraw [name=outer,  fill=aaltoBlue, fill opacity=0.1 ] (-7,-7.5) rectangle (7,2.5);
\filldraw [name=inner, fill=white, fill opacity=0.5 ] (-3,-4.5) rectangle (3,-1.5);
%\fillbetween [of=inner, fill=aaltoGreen, fill opacity=0.3 ] (-3,-4.5) rectangle (3,-1.5);
\end{tikzpicture}
 

     \tellipse{-2.5cm}{4.5cm}{3cm}{1.5cm}
 }
\end{frame}

%%%%%%%%%%%%%%%%%%%%%%%%%%%%%%%%%%%%%%%%%%
\begin{frame}[t]
    \frametitle{Modular BAG design environment principles}
        \begin{minipage}{.68\linewidth}
            \begin{itemize}
                \item Virtuoso design environment and technology definitions
                    BAG independent.
                \item BAG configuration and modules added as git modules.
                \item BAG Configuration files created by  configure script 
                \item Software paths defined by sourceme.csh 
                \item Design specific information imported with
                    design specific configure scripts or as a part of the design procedure.
            \end{itemize}
        \end{minipage}
        \begin{minipage}{.28\linewidth}
    \resizebox{\linewidth}{!}{%
\begin{tikzpicture}[every node/.style={inner sep=1ex,outer sep=0}]
\tikzstyle{dir}=[draw=black, rounded corners, 
                fill=secondarycolor!10, drop shadow,
                align=center, anchor=west, text=black, rectangle, minimum
                width=2cm,
            minimum height=0.5cm] 
\tikzstyle{file}=[ align=left, anchor=west, text=black, rectangle, minimum
                width=2cm,
            minimum height=0.5cm] 
            \node[anchor=west](root)
            [dir]{\textbf{./virtuoso\_<process>\_<version>\_<user>}};
            \node[below right = 1cm and 0.5cm of root.west ] (f1)[file] {./configure};
            \node[below = 0.5cm of f1.west ]  (f2)[file]
            {./sourceme.csh};
            \node[below = 0.5cm of f2.west ]  (f3)[file]
            {./.cdsinit};
            \node[below = 0.5cm of f3.west ]  (f4)[file] {./cds.lib};
            \node[below = 0.5cm of f4.west ]  (f5)[file]
            {./bag\_libs.def};
            \node[below = 0.5cm of f5.west ]  (f6)[file]
            {./bag\_config.yaml};
            \node[below left = 1cm and -0.5cm of f6.west ](BAGtech)
            [dir]{\textbf{Bag\_technology\_definition}};
            \node[below = 1cm of BAGtech.west ](frame)
            [dir]{\textbf{BAG\_framework}};
            \node[below = 1cm of frame.west ](templ)
            [dir]{\textbf{BAG2\_TEMPLATES\_EC}};
            \node[below = 1cm of templ.west ](bagecd)
            [dir]{\textbf{bag\_ecd}};
            \node[below = 1cm of bagecd.west ](design)
            [dir]{\textbf{diff\_amp}};
            \draw[>=latex,primarycolor, rounded corners, very thick,->] (root.west) |- (f1.west);
            \draw[>=latex,primarycolor, rounded corners, very thick,->] (root.west) |- (f2.west);
            \draw[>=latex,primarycolor, rounded corners, very thick,->] (root.west) |- (f3.west);
            \draw[>=latex,primarycolor, rounded corners, very thick,->] (root.west) |- (f4.west);
            \draw[>=latex,primarycolor, rounded corners, very thick,->] (root.west) |- (f5.west);
            \draw[>=latex,primarycolor, rounded corners, very thick,->] (root.west) |- (f6.west);
            \draw[>=latex,primarycolor, rounded corners, very thick,->]
        (root.west) |- (BAGtech.west);
            \draw[>=latex,primarycolor, rounded corners, very thick,->]
        (root.west) |- (frame.west);
            \draw[>=latex,primarycolor, rounded corners, very thick,->]
        (root.west) |- (templ.west);
            \draw[>=latex,primarycolor, rounded corners, very thick,->]
        (root.west) |- (design.west);
            \draw[>=latex,primarycolor, rounded corners, very thick,->]
        (root.west) |- (bagecd.west);
\end{tikzpicture}
 }
        \end{minipage}
\end{frame}

%%%%%%%%%%%%%%%%%%%%%%%%%%%%%%%%%%%%%%%%%%
\begin{frame}[t]
    \frametitle{Essential BAG modules}
    \begin{minipage}{0.68\linewidth}
       \begin{itemize}
           \item BAG\_framework and BAG2\_TEMPLATES\_EC 
               \begin{itemize}
                   \item Forked from \emph{https://github.com/ucb-art}. 
                   \item Some bugfixes filed as PR.
               \end{itemize}
           \item BAG\_technology\_definition 
               \begin{itemize}
                   \item In-house collection of \emph{all} technology
                       dependent data and BAG virtuoso primitives (git
                       submodule).
               \end{itemize}
       \end{itemize}
    \end{minipage}
    \begin{minipage}[t]{0.28\linewidth}
    \resizebox{\linewidth}{!}{%
\begin{tikzpicture}[every node/.style={inner sep=1ex,outer sep=0}]
\tikzstyle{dir}=[draw=black, rounded corners, 
                fill=secondarycolor!10, drop shadow,
                align=center, anchor=west, text=black, rectangle, minimum
                width=2cm,
            minimum height=0.5cm] 
\tikzstyle{file}=[ align=left, anchor=west, text=black, rectangle, minimum
                width=2cm,
            minimum height=0.5cm] 
            \node[anchor=west](root)
            [dir]{\textbf{./virtuoso\_<process>\_<version>\_<user>}};
            \node[below right = 1cm and 0.5cm of root.west ](frame)
            [dir]{\textbf{BAG\_framework}};
            \node[below = 1cm of frame.west ](templ)
            [dir]{\textbf{BAG2\_TEMPLATES\_EC}};
            \node[below = 1cm of templ.west ](BAGtech)
            [dir]{\textbf{Bag\_technology\_definition}};
            \draw[>=latex,primarycolor, rounded corners, very thick,->]
        (root.west) |- (BAGtech.west);
            \draw[>=latex,primarycolor, rounded corners, very thick,->]
        (root.west) |- (frame.west);
            \draw[>=latex,primarycolor, rounded corners, very thick,->]
        (root.west) |- (templ.west);
\end{tikzpicture}
 }
    \end{minipage}
\end{frame}
    
%%%%%%%%%%%%%%%%%%%%%%%%%%%%%%%%%%%%%%%%%%
\begin{frame}[t]
    \frametitle{BAG technology definition module}
    \begin{minipage}{.68\linewidth}
            \begin{itemize}
                \item BAG\_prim contain the Virtuoso primitives.
                \item tech\_config.yaml referenced in bag\_config.yaml
                \item tech\_params.yaml referenced in
                    BAG\_technology\_definition/\_\_init\_\_.py
                    \vfill
            \end{itemize}
        \end{minipage}
        \begin{minipage}{.28\linewidth}
    \resizebox{\linewidth}{!}{%
\begin{tikzpicture}[every node/.style={inner sep=1ex,outer sep=0}]
\tikzstyle{dir}=[draw=black, rounded corners, 
                fill=secondarycolor!10, drop shadow,
                align=center, anchor=west, text=black, rectangle, minimum
                width=2cm,
            minimum height=0.5cm] 
\tikzstyle{file}=[ align=left, anchor=west, text=black, rectangle, minimum
                width=2cm,
            minimum height=0.5cm] 
            \node[anchor=west](root)
            [dir]{\textbf{Bag\_technology\_definition}};
            \node[anchor=west, below right = 1.0 cm and 0.5cm of root.west](prim)
            [dir]{\textbf{BAG\_prim}};
            \node[anchor=west, below right = 1.0 cm and 0.0cm of
            prim.west](techdef)
            [dir]{\textbf{BAG\_technology\_definition}};
            \node[anchor=west, below right = 1.0 cm and 0.5cm of
            techdef.west](initt)
            [file]{\_\_init\_\_.py};
            \node[anchor=west, below right = 0.5 cm and 0.0cm of
            initt.west](techyaml)
            [file]{tech\_config.yaml};
            \node[anchor=west, below right = 0.5 cm and 0.0cm of
            techyaml.west](techparams)
            [file]{tech\_params.yaml};
            \node[anchor=west, below right = 0.5 cm and 0.0cm of
            techparams.west](techpy)
            [file]{tech.py};
            \node[anchor=west, below right = 1.0 cm and -0.5cm of
            techpy.west](techprim)
            [dir]{\textbf{Technology\_primitives}};
            \node[anchor=west, below right = 1.0 cm and 0.5cm of
            techprim.west](techprimbag)
            [dir]{\textbf{BAG\_prim}};
            \node[anchor=west, below right = 1.0 cm and 0.5cm of
        techprimbag.west](init)
            [file]{\_\_init\_\_.py};
            \node[anchor=west, below right = 0.5 cm and 0.0cm of
            init.west](techprimpy)
            [file]{nmos4\_fast.py};
            \node[anchor=west, below right = 0.5 cm and 0.0cm of
            techprimpy.west](techprimpy2)
            [file]{$\vdots$};
            \draw[>=latex,primarycolor, rounded corners, very thick,->]
        (root.west) |- (prim.west);
            \draw[>=latex,primarycolor, rounded corners, very thick,->]
        (root.west) |- (techdef.west);
            \draw[>=latex,primarycolor, rounded corners, very thick,->]
        (techdef.west) |- (initt.west);
            \draw[>=latex,primarycolor, rounded corners, very thick,->]
        (techdef.west) |- (techyaml.west);
            \draw[>=latex,primarycolor, rounded corners, very thick,->]
        (techdef.west) |- (techparams.west);
            \draw[>=latex,primarycolor, rounded corners, very thick,->]
        (techdef.west) |- (techpy.west);
            \draw[>=latex,primarycolor, rounded corners, very thick,->]
        (root.west) |- (techprim.west);
            \draw[>=latex,primarycolor, rounded corners, very thick,->]
        (techprim.west) |- (techprimbag.west);
            \draw[>=latex,primarycolor, rounded corners, very thick,->]
        (techprimbag.west) |- (init.west);
            \draw[>=latex,primarycolor, rounded corners, very thick,->]
        (techprimbag.west) |- (techprimpy.west);
            \draw[>=latex,primarycolor, rounded corners, very thick,->]
        (techprimbag.west) |- (techprimpy2.west);
\end{tikzpicture}
}
 \end{minipage}

\end{frame}

%%%%%%%%%%%%%%%%%%%%%%%%%%%%%%%%%%%%%%%%%%
\begin{frame}[t]
    \frametitle{BAG design generator module}
    \begin{minipage}{.68\linewidth}
            \begin{itemize}
                \item Python package
                    \begin{itemize}
                        \item Module and class layout layout generation code (for this design only). 
                        \item Module and class schematic for schematic generation code. 
                        \item Information import handled by bag\_ecd class
                        \item Generator calls defined in bag\_ecd.bag\_design
                            class.
                    \end{itemize}
            \end{itemize}
        \end{minipage}
        \begin{minipage}{.28\linewidth}
    \resizebox{\linewidth}{!}{%
\begin{tikzpicture}[every node/.style={inner sep=1ex,outer sep=0}]
\tikzstyle{dir}=[draw=black, rounded corners, 
                fill=secondarycolor!10, drop shadow,
                align=center, anchor=west, text=black, rectangle, minimum
                width=2cm,
            minimum height=0.5cm] 
\tikzstyle{file}=[ align=left, anchor=west, text=black, rectangle, minimum
                width=2cm,
            minimum height=0.5cm] 
            \node[anchor=west](root)
            [dir]{\textbf{<module\_name}};
            \node[anchor=west, below right = 1.0 cm and 0.5cm of
            root.west](templates)
            [dir]{\textbf{<module>\_templates}};
            \node[anchor=west, below right = 1.0 cm and 0.0cm of
            prim.west](module)
            [dir]{\textbf{<module\_name>}};
            \node[anchor=west, below right = 1.0 cm and 0.5cm of
            techdef.west](initt)
            [file]{\_\_init\_\_.py};
            \node[anchor=west, below right = 0.5 cm and 0.0cm of
            initt.west](layout)
            [file]{layout.py};
            \node[anchor=west, below right = 0.5 cm and 0.0cm of
            layout.west](schematic)
            [file]{schematic.py};
            \node[anchor=west, below right = 1.0 cm and -0.5cm of
            techpy.west](testbenches)
            [dir]{\textbf{(<module\_testbenches)}};
            \draw[>=latex,primarycolor, rounded corners, very thick,->]
        (root.west) |- (templates.west);
            \draw[>=latex,primarycolor, rounded corners, very thick,->]
        (root.west) |- (module.west);
            \draw[>=latex,primarycolor, rounded corners, very thick,->]
        (module.west) |- (initt.west);
            \draw[>=latex,primarycolor, rounded corners, very thick,->]
        (module.west) |- (layout.west);
            \draw[>=latex,primarycolor, rounded corners, very thick,->]
        (module.west) |- (schematic.west);
            \draw[>=latex,primarycolor, rounded corners, very thick,->]
        (root.west) |- (testbenches.west);
\end{tikzpicture}
}
 \end{minipage}

\end{frame}

%%%%%%%%%%%%%%%%%%%%%%%%%%%%%%%%%%%%%%%%%%
\begin{frame}[t]
    \frametitle{BAG ECD design structure definition module}
    \begin{minipage}{.68\linewidth}
            \begin{itemize}
                \item Python package
                    \begin{itemize}
                        \item Bag\_ecd class as parent adds the essential bag
                            modules and all generator modules to PYTHONPATH
                            automatically during generator invocation.
                        \item Bag\_ecd.bag\_design class:
                            \begin{itemize}
                                \item Defines common parameters for all
                                    designs.
                                \item Defines a method \emph{import\_design}  for importing the template
                                    information to python environment.
                                \item Defines method \emph{generate} that
                                    imports the design and generates the
                                    schematic and layout with help of methods
                                    defines layout
                                    and schematic classes of the design.
                            \end{itemize}
                    \end{itemize}
            \end{itemize}
        \end{minipage}
        \begin{minipage}{.28\linewidth}
    \resizebox{\linewidth}{!}{%
\begin{tikzpicture}[every node/.style={inner sep=1ex,outer sep=0}]
\tikzstyle{dir}=[draw=black, rounded corners, 
                fill=secondarycolor!10, drop shadow,
                align=center, anchor=west, text=black, rectangle, minimum
                width=2cm,
            minimum height=0.5cm] 
\tikzstyle{file}=[ align=left, anchor=west, text=black, rectangle, minimum
                width=2cm,
            minimum height=0.5cm] 
            \node[anchor=west](root)
            [dir]{\textbf{bag\_ecd}};
            \node[anchor=west, below right = 1.0 cm and 0.5cm of
            root.west](module)
            [dir]{\textbf{bag\_ecd}};
            \node[anchor=west, below right = 1.0 cm and 0.5cm of
            module.west](initt)
            [file]{\_\_init\_\_.py};
            \node[anchor=west, below right = 0.5 cm and 0.0cm of
            initt.west](design)
            [file]{bag\_design.py};
            \draw[>=latex,primarycolor, rounded corners, very thick,->]
        (root.west) |- (module.west);
            \draw[>=latex,primarycolor, rounded corners, very thick,->]
        (module.west) |- (initt.west);
            \draw[>=latex,primarycolor, rounded corners, very thick,->]
        (module.west) |- (design.west);
%            \draw[>=latex,primarycolor, rounded corners, very thick,->]
%        (module.west) |- (schematic.west);
%            \draw[>=latex,primarycolor, rounded corners, very thick,->]
%        (root.west) |- (testbenches.west);
\end{tikzpicture}
}
 \end{minipage}

\end{frame}

%%%%%%%%%%%%%%%%%%%%%%%%%%%%%%%%%%%%%%%%%%
\begin{frame}[t]
    \frametitle{Further actions}
    \begin{itemize}
        \item  A generic BAG\_technology\_definition  (Cadence gpdk) with ECD-compatible structure
        \item  A process generic example design using bag\_ecd structure
            virtuoso setup for Cadence gpdk
        \item Documentation with docstrings
        \item If willing to contribute, contact
            \\\emph{marko.kosunen@aalto.fi} ,\\
            or file an issue.
    \end{itemize}
\end{frame}
\end{document}
